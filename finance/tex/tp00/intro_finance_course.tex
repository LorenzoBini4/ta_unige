\documentclass[handout]{beamer}

\usepackage{ulem}
\usepackage{hyperref}
\usepackage{graphicx} % Required for inserting images
\usepackage{url}
\usepackage{amsmath}
\usepackage{amsfonts}
\usepackage{biblatex}

% Packages:
\usepackage{tabularx} % for making tables with fixed width columns
\usepackage{array} % tabularx requires this
\usepackage{amsmath} % for math
\usepackage[]{hyperref} % for links, metadata and bookmarks
\hypersetup{
    colorlinks=true,
    linkcolor=green,     % Colore dei link interni
    urlcolor=lightgreen        % Colore dei link esterni (come quello della homepage)
}
% custom light yellow color
\definecolor{lightyellow}{RGB}{255, 255, 150} 
\definecolor{lightblue}{RGB}{173, 216, 230}
\definecolor{lightred}{RGB}{255, 182, 193} 
\definecolor{lightgreen}{RGB}{144, 238, 144}

\usepackage[pscoord]{eso-pic} % for floating text on the page
\usepackage{calc} % for calculating lengths
\usepackage{bookmark} % for bookmarks
\usepackage{lastpage} % for getting the total number of pages
\usepackage{changepage} % for one column entries (adjustwidth environment)
\usepackage{paracol} % for two and three column entries
\usepackage{ifthen} % for conditional statements
\usepackage{needspace} % for avoiding page brake right after the section title
\usepackage{iftex} % check if engine is pdflatex, xetex or luatex
\usepackage{fancyhdr} % For custom headers and footers
\usepackage{lastpage} % To get the total number of pages

\title{Introduction to Computational Finance}
\subtitle{Presentation of Exercise Sessions}
\author{Lorenzo Bini}
\institute{University of Geneva}
\date{February 25, 2025}

\begin{document}

\begin{frame}
\titlepage
\end{frame}

\begin{frame}
    More about me: \pause
    \begin{itemize}
        \item Bachelor: Alma Mater Studiorum - University of Bologna (oldest university in Europe, 1088, bachelor in Physics). \pause
        \item Then Master at Polytechnic of Turin, Physics of Complex Systems. \pause
        \item Since then, research in machine learnings: \pause
        \begin{itemize}
            \item Graph Neural Networks (GNNs) for detecting MRD of Acute Lymphoblastic and Myeloid Leukemia cells classification (FCS data from HUG). \pause
            \item Self-Supervised Learning on Graphs (GCL), Geometric Deep Learning, Generative AI  (e.g., DDPMs) for 3D genomics and LLMs for medical applications (more \href{https://lorenzobini4.github.io/}{here}). \pause
            \item PhD in graph machine learning with Prof. Stéphane Marchand-Maillet (almost 2 year and a half now). \pause
        \end{itemize}
    \end{itemize}
    \vspace{1cm}
    Quick remarks: \pause
    \begin{itemize}
        \item I'm discovering Finance with you! \pause
        \item I mainly code in Python/Julia.
    \end{itemize}
\end{frame}

\begin{frame}
    Instructions: \pause
    \begin{itemize}
    \item Submit a \textbf{zip} file containing your report in \textbf{pdf} format and your code. \pause
    \item \textbf{Deadline}: Every Monday night at 11:59 PM. \pause
    \item Report: \pause
    \begin{itemize}
    \item Answers to the TPs questions. Figures and numerical results are necessary but not sufficient: provide your analysis/comments as well. \pause
    \item Example: if your results surprise you, explain why and explain what the expected result was. \pause
    \item Pay attention to the presentation: axis labels, legend, etc. \pause
    \end{itemize}
    \item Code: \pause
    \begin{itemize}
        \item Implementation questions + code used to produce the results and figures in the report. \pause
        \item Programming language of your choice. \pause
        \item I recommend Python/Julia: I won't be able to assist you in the same way if you use a different language.
    \end{itemize}
\end{itemize}
\end{frame}

\begin{frame}
Session organization: \pause
\vspace{1cm}
\begin{enumerate}
\item Answers to your questions from the previous session. \pause
\item Comments on the TPs I've just corrected. \pause
\item Presentation of current TP.
\end{enumerate}
\end{frame}

\begin{frame}
Questions during the week: \pause
\vspace{1cm}
\begin{itemize}
\item By \textbf{email} (\href{mailto:Lorenzo.Bini@unige.ch}{Lorenzo.Bini@unige.ch}) rather than via Zoom or we can arrange an appointment at my office (Bureau \#221). \pause
\item Any question is good, don't hesitate to contribute to the answer if you think you can help.
\end{itemize}
\end{frame}

\begin{frame}
\textbf{Important reminders}: \pause
\vspace{1cm}
\begin{itemize}
\item TPs will be graded on binary scale \textit{passed or not}, where you must complete (graded as \textit{passed}) \textbf{8 TPs out of 13} to be eligible for the final oral exam. \pause
\item Fill out the "What about you" form on Moodle (just couple of minutes, to better know your background and knowledge).
\end{itemize}
\end{frame}



\end{document}