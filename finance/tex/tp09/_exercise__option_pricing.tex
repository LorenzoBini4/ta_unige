\section*{Option pricing: Black-Scholes versus Binomial Tree}

Let an asset $S$ be valuated at $t=0$ at $S_0=100$. We consider an european option to buy this asset (call) with maturity $T = 1$ (in years) and strike price $K = 120$. The goal of this exercise is to compare two methods to price this option, Black-Scholes and Binomial Tree. We assume a constant volatility $\sigma=20\%$ over the lifespan of the call, and a risk-free rate $r=5\%$. 

\begin{enumerate}
    \item Implement the Black-Scholes formula to determine the value of this call at $t=0$.
    \item Implement a binomial tree to determine the initial value of the call. Your implementation should take the depth of the tree as an argument.
    \item On the same graph, plot the evolution of the estimated value of the call option as a function of the binomial tree depth, as well as the value derived with Black-Scholes. What do you observe? How deep should be the tree in order to get a reasonable approximation of the Black-Scholes value?
\end{enumerate}

