\section*{The Greeks}
Let an asset $S$ with initial value $S_0$ at $t=0$. We consider both a European option to buy this asset (call) and to sell it (put), both with maturity $T = 1$ (in years) and strike price $K = 120$. We assume a constant volatility $\sigma=20\%$ over their lifespan, and a risk-free rate $r=5\%$. 

\subsection*{Plotting the greeks}
Using the Black-Scholes formula to value the call and the put price:
As a function of the initial asset price $S_0$, plot the evolution of:
\begin{itemize}
    \item the price of the call/put
    \item the $\Delta$ of the call/put
    \item the $\Gamma$ of the call/put
\end{itemize}

\noindent Comment on the meaning of these graphs.

\subsection*{Hedging a short call}
Suppose the asset price is $S_0 = 100$, and we sell $1000$ calls. 
\begin{itemize}
    \item Which position should we have to be $\Delta$-neutral?
    \item What is the profit of this strategy, supposing that $S_{0+\epsilon} = 105$? $S_{0+\epsilon} = 95$? 
    \item In the two previous cases, compare the quality of this hedging strategy with having a fully naked (no position) or a fully covered option (buying a quantity of 1000 of the underlying asset).
\end{itemize}

\subsection*{Hedging a short put}
Suppose the asset price is $S_0 = 100$, and we sell $1000$ puts. 
\begin{itemize}
    \item Which position should we have to be $\Delta$-neutral?
    \item What is the profit of this strategy, supposing that $S_{0+\epsilon} = 105$? $S_{0+\epsilon} = 95$? 
    \item In the two previous cases, compare the quality of this hedging strategy with having a fully naked or a fully covered option.
\end{itemize}

%%%%%%%%%%%%%%%%%%%%%%%%%%%%%%%%%%%%%%%%%%%%%%%%%%%%%%%%%%%%%%%%%%%%%%%%%%%%%%%
