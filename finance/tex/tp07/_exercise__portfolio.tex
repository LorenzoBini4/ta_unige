\section*{Optimal Portfolio}
In this series we look at the closing prices of Macdonald's, Bank of America, IBM, Chevron, Coca-Cola, Novartis and AT\&T, over a one-year time span extending from 2013-05-01 to 2014-05-01. \\

\noindent Download the file \texttt{closing\_prices.csv} on Moodle. You can load it with the following Python code:

\begin{minted}{python}
import pandas as pd
df = pd.read_csv('closing_prices.csv')
data = df.to_numpy() # if you prefer working with np array
\end{minted}

\noindent We define the weight vector of the portfolio $w = \{w_1, ..., w_7\}$ such that $\sum\limits_{i=1}^{7} w_i = 1$.

\begin{enumerate}
    \item Write a function that estimates the expected return and the risk (standard deviation of returns) for a given weight vector. Then, plot the return against the risk for 100000 randomly chosen weights vectors (Monte-Carlo simulation). What do you observe?
    \item We introduced the following analytical expressions during the course to compute the weight vector that minimizes the risk given a desired portfolio return $\mu_p$:
        \begin{equation}
        \begin{aligned}
            a &= \mathbf{1}^T C^{-1} \mathbf{1} \\
            b &= \mathbf{1}^T C^{-1} \mu  \\
            c &= \mu^T C^{-1} \mu  \\
            d &= ac - b^2 \\
            \lambda_1 &= \frac{c - b \mu_p}{d} \\
            \lambda_2 &= \frac{a \mu_p - b}{d} \\
            w &= C^{-1} (\lambda_1 \mathbf{1} + \lambda_2 \mu)  \\
        \end{aligned}
        \end{equation}
    With $\mu = \{\mu_1, ..., \mu_7\}$ the expected returns of each stock and $C$ the covariance matrix of returns.\\ 
        \noindent Using this expression, draw Markowitz's efficient frontier for portfolio return $\mu_p \in [-0.0006, +0.0004]$.
    \item Using the efficient frontier, find the weight of the portfolio with the minimal volatility. What can you say about the return of this portfolio?
\end{enumerate}

%%%%%%%%%%%%%%%%%%%%%%%%%%%%%%%%%%%%%%%%%%%%%%%%%%%%%%%%%%%%%%%%%%%%%%%%%%%%%%%
\newpage
\section*{Optimal Portfolio under No-Short Selling Constraints}
At this point, you will extend the optimal portfolio analysis by imposing a no-short selling constraint (i.e. all portfolio weights must be non-negative).
You will compute the efficient frontier under this additional constraint and compare it to the unconstrained (Markowitz) efficient frontier.

\noindent Keep using the provided \texttt{closing\_prices.csv} file which contains the closing prices of Mcdonald's, Bank of America, IBM, Chevron, Coca-Cola, Novartis, and AT\&T over one year (from 2013-05-01 to 2014-05-01). Load the data and compute the daily returns for each stock.

\begin{enumerate}
        \item Formulate the portfolio optimization as a minimization problem where the objective is to minimize the portfolio variance subject to the following constraints:
        \begin{itemize}
            \item The portfolio return is equal to a given target $\mu_p$.
            \item The sum of weights is equal to 1.
            \item All weights are non-negative (no short selling).
        \end{itemize}
        \item Use a suitable Python optimization library (e.g., \texttt{cvxpy}) to solve the constrained problem.
        \item Plot the constrained efficient frontier over the same range of previous $\mu_p$.
        \item Compare the unconstrained (computed from Exercise \# 1) and constrained efficient frontiers.
        \item Discuss the impact of the no-short selling constraint on the risk-return trade-off.
        \item Identify and report the portfolio with minimal risk under the no-short selling constraint and comment on its expected return.
\end{enumerate}
\textbf{Note:} You can use the \texttt{cvxpy} library to solve the constrained optimization problem. The library allows you to define the optimization variables, objective function, and constraints in a straightforward manner. Make sure to install it if you haven't already by \texttt{pip install cvxpy}.
