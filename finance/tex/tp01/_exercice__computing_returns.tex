\section{Computing returns}

The following series describes the value of a portfolio at the end of each month:\\

\begin{tabular}{|c|ccccccccccccc|}
\hline
Month &    & Jan & Feb & Mar & Apr & May & Jun & Jul & Aug & Sep & Oct & Nov & Dec \\ 
\hline
t     & 0 & 1 & 2 & 3 & 4 & 5 & 6 & 7 & 8 & 9 & 10 & 11 & 12 \\
\hline
P(t)  & 100 & 101 & 102 & 103 & 104 & 105 & 106 & 107 & 108 & 109 & 110 & 111 & 112 \\
\hline
\end{tabular}
\\

\noindent The initial value of the portfolio being 100, the annual return is : $$R_a = R_{12}(12) = \frac{P(12) - P(0)}{P(0)} = 12\%$$

\begin{enumerate}
    \item Compute monthly returns $\{R(t) \mid t = 1,...,12\}$.
    \item Compute the annual return $R_a$ from the monthly returns (you should find $12\%$ again). Compare it to the sum of the monthly returns.  
    \item Compute the average monthly return $R_m$ and compare it to the average of the monthly returns.
\end{enumerate}

